\documentclass[12pt, letterpaper, twoside]{article}

\usepackage{amsmath}

\title{Apuntes de Ondas}
\author{Ángel Ruiz Fernández B2A}
\date{Diciembre 2022}

\usepackage[a4paper, total={6in, 9in}]{geometry}

\newcommand{\dd}[1]{\mathrm{d}#1}

\begin{document}
	\maketitle
	
	\begin{equation}
		x_{rms} = \frac{x}{\sqrt{2}}
	\end{equation}
	
	Relación entre capo electrico y magnetico. $v$ velocidad de propagación. 
	
	\begin{equation}
		\hat{s} \times \vec{B} = - \frac{1}{v} \vec{E}
	\end{equation}

	\begin{equation}
		\hat{s} \times \vec{E} = v \vec{B}
	\end{equation}

	\begin{equation}
		|\vec{E}| = c|\vec{B}|
	\end{equation}
	\begin{equation}
		|\vec{B}| = \frac{|\vec{E}|}{c}
	\end{equation}

	Vector de Poynting
	\begin{equation}
		\vec{S} = \vec{H} \times \vec{B}
	\end{equation}

	\begin{equation}
		\vec{S} = \vec{E} \times \vec{H}
	\end{equation}

	Valor promedio de Poynting. $P$ potencia, $S$ superficie
	\begin{equation}
		<S> = \frac{P}{S}
	\end{equation}
	\begin{equation}
		<S> = \frac{1}{2} \varepsilon_0 c E^2_0
	\end{equation}

	Energía
	\begin{equation}
		E_0 = \sqrt{\frac{2<S>}{\varepsilon_0 c}}
	\end{equation}

	\begin{equation}
		W = \epsilon E^2  = \frac{B^2}{\mu}
	\end{equation}

	Vector de onda, $s$ es dirección
	\begin{equation}
		\vec{k} = \frac{2 \pi}{\lambda} \hat{s}
	\end{equation}

	tomar nota: mancha de Airy
	
	\section{Optica Geometrica}
	
	Fermat.
	
	I las trayectorias en los medios homogeneos e isotropos son rectos
	
	II. El rayo incidente, el refractado o reflejado y la normal estan en un mismo plano (de incidencia).
	
	III. Snell
	\begin{equation}
		n sin \varepsilon = n' sin \varepsilon'
	\end{equation}

	Reflexión: 
	\begin{equation}
		\varepsilon = \varepsilon'
	\end{equation}
	
	IOR aire $n = 1$

	Invariante de Abbe
	\begin{equation}
		n ( \frac{1}{r} - \frac{1}{s} ) = n' (\frac{1}{r} - \frac{1}{s'} )
	\end{equation}

	Formula general | fabricante de lentes	
	\begin{equation}
		\frac{n}{f'} = (n_{lente} - n)(\frac{1}{r_1} - \frac{1}{r_2})
	\end{equation}

	Espejo
	\begin{equation}
		\frac{n}{s} + \frac{n'}{s'} = \frac{n'}{f'}
	\end{equation}

	Lente
	\begin{equation}
		-\frac{n}{s} + \frac{n'}{s'} = \frac{n'}{f'}
	\end{equation}

	\begin{equation}
		r = 2f'
	\end{equation}

	\begin{equation}
		\pm \frac{1}{s} + \frac{1}{s'} = \frac{2}{r}
	\end{equation}
	
	Potencia en dioptrías D
	\begin{equation}
		P = \frac{1}{f}
	\end{equation}
	
	Aumento lateral
	\begin{equation}
		\beta = \frac{y'}{y} = \frac{n'}{n} \frac{s'}{s}
	\end{equation}

	Aumento visual comercial (distancia de observación | punto proximo 250mm)
	\begin{equation}
		\Gamma' = \frac{250}{f'}
	\end{equation}

	Limite de Dawes | poder resolutivo $D$ es diametro de lente (apertura)
	\begin{equation}
		\theta = \frac{1.22 \lambda}{D} \approx \frac{115}{D (mm)}
	\end{equation}
	
	
\end{document}