\documentclass[12pt, letterpaper, twoside]{article}

\usepackage{amsmath}

\title{Apuntes de Ondas}
\author{Ángel Ruiz Fernández B2A}
\date{Diciembre 2022}

\usepackage[a4paper, total={6in, 9in}]{geometry}

\newcommand{\dd}[1]{\mathrm{d}#1}

\begin{document}
	\maketitle
	
	Ecuación de onda
	\begin{equation}
		\frac{\partial^2 \Psi}{\partial t^2} = v^2 \Delta \Psi
	\end{equation}
	
	en una dimension 
	\begin{equation}
		\frac{\partial^2 \Psi}{\partial t^2} = v^2 \frac{\partial^2 \Psi}{\partial x^2}
	\end{equation}
	
	\begin{equation}
		y(x) = f(x)
	\end{equation}
	
	\begin{equation}
		y(x, t) = f(x + vt) = f(kx + vt)
	\end{equation}

La amplitud es el maximo valor que puede alcanzar

Oscilador armónico simple
	\begin{equation}
		y(x, t) = A sin(kx + \omega t + \phi_0)
	\end{equation}
	$A$ es amplitud, $k$ numero de ondas que hay en $2\pi$, $\omega$ velocidad de propagación
	
	\begin{equation}
		k = \frac{2\pi}{\lambda}
	\end{equation}

	\begin{equation}
		\omega = 2\pi f
	\end{equation}

	\begin{equation}
		v = \frac{\lambda}{T} = \lambda f = \frac{\omega}{k}
	\end{equation}

	\begin{equation}
		\lambda = \frac{v}{f}
	\end{equation}

	Velocidad de propagación que para ondas mecanicas 
	\begin{equation}
		v = \sqrt{\frac{p_{elastica}}{p_{inercial}}} = \sqrt{\frac{T}{\mu}}
	\end{equation}
	Freq de armonico $n$ en cuerda de longitud $l$ y v de prop
	\begin{equation}
		f = \frac{nv}{2l}
	\end{equation}
	
	Energía
	\begin{equation}
		\Delta E = \frac{1}{2} \mu \cdot \Delta x \cdot \omega^2 \cdot A^2 = 2 \mu \cdot \Delta x \cdot \pi^2 \cdot f^2 \cdot A^2
	\end{equation}

	Potencia
	\begin{equation}
		P = \frac{\Delta E}{\Delta t} = 2 \mu \cdot \pi^2 \cdot f^2 \cdot A^2 \cdot v
	\end{equation}
	
	Intensidad
	\begin{equation}
		I = \frac{P}{S}
	\end{equation}
	De onda esferica
	\begin{equation}
		I = \frac{P}{4 \pi r^2}
	\end{equation}
	
	\section{Sonido}
	La velocidad de propagación del sonido en el aire a 15ºC son $340m/s$.
	Umbral de audición $I_0 = 10^{-12} W/m^2$ y umbral de dolor humano $I_1 = 1W/m^2$.
	
	Decibelio, potencia relacionado con nivel de intensidad
	\begin{equation}
		\beta = 10 log (\frac{I}{I_0})
	\end{equation}
	\begin{equation}
		I = I_0 \cdot 10^{\frac{\beta}{10}}
	\end{equation}

	Velocidad del sonido en el aire
	\begin{equation}
		v = \sqrt{\frac{\gamma R T}{M}}
	\end{equation}

	Efecto doppler
	\begin{equation}
		f' = f \frac{v \pm v_0}{v \mp v_F}
	\end{equation}
\end{document}