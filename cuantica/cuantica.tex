\documentclass[12pt, letterpaper, twoside]{article}

\usepackage{amsmath}
\usepackage{siunitx}

\title{Apuntes de Cuantica}
\author{Ángel Ruiz Fernández B2A}
\date{Marzo 2023}

\usepackage[a4paper, total={6in, 9in}]{geometry}

\newcommand{\dd}[1]{\mathrm{d}#1}

\begin{document}
	\maketitle
	
	$q_e = 1.602 \cdot 10^{-19} \si{C}$
	
	$1 \si{eV} = 1.602 \cdot 10^{-19} \si{J}$
	
	$1 \si{\AA} = 10^{-10} \si{m}$
	
	$m_e = 9.11 \cdot 10^{-31} \si{kg}$
	
	Energía en eV es la que obtiene un electron cuando pasa por un potencial de V
	\begin{equation}
		V = \frac{mv^2}{2q}
	\end{equation}
	
	\section{Cuerpo negro}
	
	Ley de Stefan-Boltzmann $\sigma = 5.67 \cdot 10^{-8} \si{W \cdot m^{-2} K^{-4}}$ $I$ instensidad, $T$ temperatura absoluta
	\begin{equation}
		I = \sigma T^4
	\end{equation}
	
	Ley de desplazamiento de Wien	
	\begin{equation}
		\lambda_{max} T = 0.2897 \si{cm \cdot K}	
	\end{equation}

	\section{Efecto fotoelectrico}
	
	Energía del fotón $h = 6.64 \cdot 10^{-34} \si{J \cdot s}$
	\begin{equation}
		E = hf
	\end{equation}

	Explicación de Einstein $\hbar = \frac{h}{2 \pi}$
	\begin{equation}
		hf = hf_0 + \frac{1}{2} m_e v^2_{max}
	\end{equation}

	La energía cinetica del electron arrancado, es la energia del foton entrante menos la umbral
	\begin{equation}
		E_{ce^-} = E_f - E_{umbral}
	\end{equation}

	\section{Espectros atomicos y Bohr}
	Rydberg y Ritz $R = 1.0967760 \cdot 10^{-7} \si{m^{-1}}$
	\begin{equation}
		\frac{1}{\lambda} = R ( \frac{1}{n_{1}^2} - \frac{1}{n^{2}_2})
	\end{equation}

	Series
	\begin{itemize}
		\item n = 1 Lyman | UV
		\item n = 2 Balmer | Visible
		\item n = 3 Paschen | IR
		\item n = 4 Brackett | IR
		\item n = 5 Pfund | IR
		\item n = 6 Humphreys | IR
	\end{itemize}

	Radio orbitas del H
	\begin{equation}
		r = 0.529 \cdot n^2 \si{\AA}
	\end{equation}

	Energía orbitas del H
	\begin{equation}
		E_{total} = - \frac{13.6}{n^2} \si{eV}
	\end{equation}

	Momento, radio y numero n
	\begin{equation}
		rp = n \hbar
	\end{equation}

	\section{Dualidad onda particula y hipotesis de De Broglie}
	
	Longitud de onda de una particula masiva
	\begin{equation}
		\lambda = \frac{h}{mv}
	\end{equation}


	\section{Funciones de onda y ecuación de Schrödinger}
	Numeros cuanticos, n es orbita, l tipo de orbital, m es orientación magnetica, s spin

	Orbitales
	\begin{itemize}
		\item l = 0 | s
		\item l = 1 | p
		\item l = 2 | d
		\item l = 3 | f
		\item l = 4 | g
	\end{itemize}

	\section{Principio de indeterminacion de Heisenberg}
	
	Incertidumbre de espacio y momento en 1 dimensión
	\begin{equation}
		\Delta x \cdot \Delta p_x \geq \frac{\hbar}{2}
	\end{equation}

	Cuerpos masivos
	\begin{equation}
		\Delta x \cdot \Delta v \geq \frac{\hbar}{2m}
	\end{equation}
\end{document}