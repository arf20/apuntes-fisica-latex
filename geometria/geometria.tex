\documentclass[12pt, letterpaper, twoside]{article}

\usepackage{amsmath}
\usepackage{siunitx}
\usepackage{multicol}
\usepackage{mathtools}

\title{Apuntes de Geometría}
\author{Ángel Ruiz Fernández B2A}
\date{Abril 2023}

\usepackage[a4paper, total={6in, 9in}]{geometry}

\newcommand{\dd}[1]{\mathrm{d}#1}

\begin{document}
	\maketitle
	
	\section{Productos}
	
	Escalar
	\begin{equation}
		\vec{u} \cdot \vec{v} = |\vec{u}| \cdot |\vec{v}| \cdot cos\alpha = \vec{v}_x \vec{u}_x + \vec{u}_y \vec{v}_y + \vec{u}_z \vec{v}_z
	\end{equation}

	Vectorial
	\begin{equation}
		\vec{u} \times \vec{v} = 
		\begin{vmatrix}
			\hat{i}   & \hat{j}   & \hat{k}   \\
			\vec{u}_x & \vec{u}_y & \vec{u}_z \\
			\vec{v}_x & \vec{v}_y & \vec{v}_z \\
		\end{vmatrix}
	\end{equation}

	Mixto
	Vectorial
	\begin{equation}
		[\vec{u}, \vec{v}, \vec{w}] = 
		\begin{vmatrix}
			\vec{u}_x & \vec{u}_y & \vec{u}_z \\
			\vec{v}_x & \vec{v}_y & \vec{v}_z \\
			\vec{w}_x & \vec{w}_y & \vec{w}_z \\
		\end{vmatrix}
	\end{equation}
	
	\section{Ecuaciones de la recta}
	
	Sea un punto $P$ y un vector director $\vec{d}$, la recta $r$:
	
	Paramétrica
	\begin{equation}
		r :
		\begin{cases}
			x = P_x + \vec{d}_x \lambda \\
			y = P_y + \vec{d}_y \lambda \\
			z = P_z + \vec{d}_z \lambda \\
		\end{cases}
	\end{equation}

	Punto genérico
	\begin{equation}
		G(\lambda) = (P_x + \vec{d}_x \lambda, P_y + \vec{d}_y \lambda, P_z + \vec{d}_z \lambda)
	\end{equation}
	
	Vectorial
	\begin{equation}
		R(\lambda) = P + \vec{d} \lambda
	\end{equation}

	Continua
	\begin{equation}
		\frac{x - P_x}{\vec{d}_x} = \frac{y - P_y}{\vec{d}_y} = \frac{z - P_z}{\vec{d}_z}
	\end{equation}

	General
	\begin{equation}
		r :
		\begin{cases}
			A_1x + B_1y + C_1z  D_1 = 0 \\
			A_2x + B_2y + C_2z  D_2 = 0 \\
		\end{cases}
	\end{equation}


	\section{Ecuaciones del plano}
	
	Sea un punto $P$ y dos vectores directores $\vec{u}$ y $\vec{v}$ o vector normal $\vec{n}$, el plano $\pi$:
	
	Con directores
	\begin{equation}
		\pi : \begin{cases}
			P \\
			\vec{u} \\
			\vec{v} \\
		\end{cases}
	\end{equation}

	Con normal
	\begin{equation}
		\pi : \begin{cases}
			P \\
			\vec{n} \\
		\end{cases}
	\end{equation}

	Paramétrica
	\begin{equation}
		\pi :
		\begin{cases}
			x = P_x + \vec{u}_x \lambda + \vec{v}_x \mu \\
			y = P_y + \vec{u}_y \lambda + \vec{v}_y \mu \\
			z = P_z + \vec{u}_z \lambda + \vec{v}_z \mu \\
		\end{cases}
	\end{equation}
	
	
	Implícita
	\begin{equation}
		\pi :
		\begin{vmatrix}
			x - P_x   & y - P_y   & z - P_z   \\
			\vec{u}_x & \vec{u}_y & \vec{u}_z \\
			\vec{v}_x & \vec{v}_y & \vec{v}_z \\
		\end{vmatrix}
	\end{equation}
	\begin{equation}
		= Ax + By + Cz + D = 0
	\end{equation}


	\section{Posiciones relativas}
	
	punto-recta
	\begin{equation}
		P = P_r + \vec{d_r} \lambda =
		\begin{cases}
			\exists     & P \in r    \\
			\not\exists & P \not\in r \\
		\end{cases}
	\end{equation}

	recta-recta
	\begin{equation}
		\vec{d_r} \cdot \vec{d_s}
		\begin{cases}
			0     & \perp     \\
			\not0 & \not\perp \\
		\end{cases}
	\end{equation}

	\begin{equation}
		\vec{d_r} \propto \vec{d_s}
		\begin{cases}
			// & P \in r
			\begin{cases}
				r \equiv s
			\end{cases}
		\end{cases}
	\end{equation}

	\begin{equation}
		rg(\vec{d_r}, \vec{d_s}, \vec{P_r P_s})
		\begin{cases}
			1     & r \equiv s     \\
			2     & rg(\vec{d_r}, \vec{d_s})
			\begin{cases}
				1     & //       \\
				2     & secantes \\
			\end{cases} \\
			3     & cruzadas
		\end{cases}
	\end{equation}
	
	punto-plano
	\begin{equation}
		A P_x + B P_y + C P_z + D = 0
		\begin{cases}
			0     & P \in \pi     \\
			\not0 & P \not\in \pi \\
		\end{cases}
	\end{equation}

	recta-plano
	\begin{multicols}{2}
		\noindent
		\begin{equation}
			A =
			\begin{pmatrix}
				A_{r_1} & B_{r_1} & C_{r_1} \\
				A_{r_2} & B_{r_2} & C_{r_2} \\
				A_\pi   & B_\pi   & C_\pi   \\
			\end{pmatrix}
		\end{equation}
		\begin{equation}
			A' =
			\begin{pmatrix}
				A_{r_1} & B_{r_1} & C_{r_1} & D_{r_1} \\
				A_{r_2} & B_{r_2} & C_{r_2} & D_{r_2} \\
				A_\pi   & B_\pi   & C_\pi   & D_\pi   \\
			\end{pmatrix}
		\end{equation}
	\end{multicols}
	\begin{equation}
		rg(A), rg(A')
		\begin{cases}
			2, 2 & \text{SCI} \quad r  \in \pi  \\
			3, 3 & \text{SCD} \quad secante     \\
			2, 3 & \text{SI}  \quad //          \\
		\end{cases}
	\end{equation}
	o
	\begin{equation}
		A (P_{r_x} + \vec{d_r}_x) \lambda + B (P_{r_y} + \vec{d_r}_y) \lambda + C (P_{r_z} + \vec{d_r}_z) \lambda + D = 0
		\begin{cases}
			0=0                 & r \in \pi   \\
			\exists\lambda      & secante     \\
			\not\exists\lambda  & //          \\
		\end{cases}
	\end{equation}
	o
	\begin{equation}
		\begin{cases}
			\vec{d_r} \cdot \vec{n_\pi} = 0 \quad \text{y} \quad P_r \in \pi       & r \in \pi   \\
			\vec{d_r} \cdot \vec{n_\pi} \not= 0                 & secante     \\
			\vec{d_r} \cdot \vec{n_\pi} = 0 \quad \text{y} \quad P_r \not\in \pi   & //          \\
		\end{cases}
	\end{equation}
	
	\section{Distancias}
	
	punto-punto
	\begin{equation}
		d(A, B) = \sqrt{\vec{AB}_{x}^2 + \vec{AB}_{y}^2 + \vec{AB}_{z}^2}
	\end{equation}

	punto-recta
	\begin{equation}
		d(P, r(A, \vec{d_r})) = \frac{|\vec{d_r} \times \vec{AP}|}{|\vec{d_r}|}
	\end{equation}

	punto-plano
	\begin{equation}
		d(P(x_0, y_0, z_0), \pi) = \frac{|Ax_0 + By_0 + Cz_0 + D|}{\sqrt{A^2 + B^2 + C^2}}
	\end{equation}

	recta-recta //
	\begin{equation}
		d(r, s) = d(P_r, s) = d(P_s, r)
	\end{equation}

	recta-recta cruzadas
	\begin{equation}
		d(r, s) = h = \frac{V}{A_b} = \frac{|[\vec{P_r P_s}, \vec{d_r}, \vec{d_s}]|}{|\vec{d_r} \times \vec{d_s}|}
	\end{equation}

	recta-plano //
	\begin{equation}
		d(r, \pi) = d(P_r, \pi)
	\end{equation}

	
	\section{Ángulos}
	
	vector-vector o recta-recta $d_r$ $d_s$
	\begin{equation}
		cos(\alpha) = \frac{|\vec{u} \cdot \vec{v}|}{|\vec{u}| \cdot |\vec{v}|}
	\end{equation}

	plano-plano (sus normales)
	\begin{equation}
		\theta(\pi_1, \pi_2) = \theta(n_{\pi_1}, n_{\pi_2})
	\end{equation}

	recta-plano
	\begin{equation}
		\theta(r, \pi) = \theta(d_r, n_\pi)
	\end{equation}


	\section{Intersecciones}
	
	\begin{itemize}
		\item recta-recta secantes:
		Igualar paramétricas
		
		\item recta-plano:
		Substituir la paramétrica de $r$ en implícita de $\pi$
		
		\item plano-plano:
		Simplemente juntas las dos implícitas y esa es la recta
	\end{itemize}

	\section{Proyecciones}
	
	\begin{itemize}
		\item punto en recta:
		Plano perpendicular a $r$ que contenga $P$: $\pi: P, \vec{d_r}$;
		Cortar $r$ con $\pi$; P'
		
		\item punto en plano:
		Perpendicular a $\pi$ que pasa por $P$: $r: P, \vec{n_\pi}$;
		Cortar $r$ con $\pi$; P'
		
		\item recta en plano:
		Proyectar dos puntos de la recta en el plano y hacer recta proyectada; 
		o Hacer plano perpendicular a $\pi$ que contenga a $r$ y intersección de planos
	\end{itemize}


	\section{Volúmenes}

	Paralelepípedo
	\begin{equation}
		V = |[\vec{AB}, \vec{AC}, \vec{AD}]|
	\end{equation}
	
	Tetraedro 
	\begin{equation}
		V = \frac{1}{6} |[\vec{AB}, \vec{AC}, \vec{AD}]|
	\end{equation}


	\section{Áreas}
	
	Paralelogramo
	\begin{equation}
		A = |\vec{AB} \times \vec{AC}|
	\end{equation}
	
	Triangulo
	\begin{equation}
		A = \frac{1}{2} |\vec{AB} \times \vec{AC}|
	\end{equation}
	
\end{document}