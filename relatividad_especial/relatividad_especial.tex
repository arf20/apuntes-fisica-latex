\documentclass[12pt, letterpaper, twoside]{article}

\usepackage{amsmath}

\title{Apuntes de Relatividad Especial}
\author{Ángel Ruiz Fernández B2A}
\date{Marzo 2023}

\usepackage[a4paper, total={6in, 9in}]{geometry}

\newcommand{\dd}[1]{\mathrm{d}#1}

\begin{document}
	\maketitle
	
	Sean dos observadores $O$ en reposo y $O'$ que se mueve a una velocidad constante $v'$ (las aceleraciones ya no es Especial).
	
	Cuando $v'$ se aproxima a $c$ ocurren cosas nazis.
	
	Factor de Lorentz
	\begin{equation}
		\gamma \equiv \frac{1}{\sqrt{1 - \frac{v^2}{c^2}}}
	\end{equation}

	\begin{equation}
		x' = \frac{x}{\gamma}
	\end{equation}
	
	\section{Contracción temporal}
	\begin{equation}
		t ' = t \sqrt{1 - \frac{v^2}{c^2}}
	\end{equation}

	\section{Contracción de longitud}
	\begin{equation}
		l ' = l \sqrt{1 - \frac{v^2}{c^2}}
	\end{equation}

	\section{Contracción de velocidad}
	\begin{equation}
		v ' = \frac{\Delta x'}{\Delta t'}
	\end{equation}

	\section{Contracción de masa}
	\begin{equation}
		F = \frac{m\Delta v}{\Delta t}
	\end{equation}
	\begin{equation}
		m = \gamma m_0
	\end{equation}
	\begin{equation}
		m' = \frac{m}{\sqrt{1 - \frac{v^2}{c^2}}}
	\end{equation}

	\section{Energía relativista}
	\begin{equation}
		F = \frac{dp'}{dt}
	\end{equation}
	\begin{equation}
		E_c = \gamma m_0 c^2 - m_0 c^2
	\end{equation}
	\begin{equation}
		\Delta E = \Delta m c^2
	\end{equation}
	
	
\end{document}