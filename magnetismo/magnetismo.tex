\documentclass[12pt, letterpaper, twoside]{article}

\usepackage{amsmath}

\title{Apuntes de Magnetismo}
\author{Ángel Ruiz Fernández B2A}
\date{Diciembre 2022}

\usepackage[a4paper, total={6in, 9in}]{geometry}

\newcommand{\dd}[1]{\mathrm{d}#1}

%\usepackage[showframe,paper=a4paper,margin=1in]{geometry}
%\setlength{\droptitle}{-0.5in}


\begin{document}
	\maketitle
	
	Producto vectorial
	\begin{equation}
		\vec{C} = \vec{A} \times \vec{B} =
		\begin{vmatrix}
			\hat{i} & \hat{j} & \hat{k}\\
			A_x & A_y & A_z\\
			B_x & B_y & B_z
		\end{vmatrix}
	\end{equation}

	\begin{equation}
		C = AB sin(\theta) \hat{u}_{A\rightarrow B}
	\end{equation}
			
	$\hat{u}_{A\rightarrow B}$ es lo de enroscar el tapón de $\vec{A}$ a $\vec{B}$
	
	\section{Interacción magnética}
	
	\subsection{Fuerza de Lorentz}
	
	\begin{equation}
		\vec{F} = q ( \vec{E} + \vec{v} \times \vec{B})
	\end{equation}

	\begin{equation}
		F = q v B sin(\theta)
	\end{equation}
	
	\subsection{Fuerza de un campo sobre una corriente}
	
	\begin{equation}
		\vec{F} = I \cdot \vec{l} \times \vec{B}
	\end{equation}
	
	\subsection{Campo magnético de corriente}
	
	\begin{equation}
		I = \dfrac{dq}{dt}
	\end{equation}
	
	\begin{equation}
		I \vec{dl} = \dfrac{dq}{dt} \vec{dl} = dq \dfrac{\vec{dl}}{dt} = dq \cdot \vec{v}
	\end{equation}
	
	\begin{equation}
		\vec{dB} = k_m \frac{I\vec{dl} \times \hat{u}_r}{r^2} = k_m \frac{dq \cdot \vec{v} \times \hat{u}_r}{r^2}
	\end{equation}
	
	Particula discreta moviendose
	\begin{equation}
		\vec{B} = \frac{\mu_0 q \cdot \vec{v} \times \hat{u}_r}{4 \pi r^2} 
	\end{equation}
	
	
	En cable recto
	\begin{equation}
		\vec{B} = \frac{\mu_0I}{2 \pi r} \hat{u}_n
	\end{equation}
	
	En espira(s)
	\begin{equation}
		\vec{B} = \frac{\mu_0NI}{2 r} \hat{u}_n
	\end{equation}

	\subsection{Orbita de particula cargada}

	\begin{equation}
		F_c = \frac{mv^2}{r}
	\end{equation}

	\begin{equation}
		r = \frac{mv}{qB}
	\end{equation}

	\begin{equation}
		\omega = \frac{v}{r}
	\end{equation}

	\begin{equation}
		T = \frac{2 \pi}{\omega} = \frac{2 \pi m}{qB}
	\end{equation}
	
	\section{Inducción electromagnética}

	Definir flujo magnético

	\begin{equation}
		\Phi_B = \iint_\Sigma \vec{B} \cdot \vec{dS}
	\end{equation}

	\begin{equation}
		\Phi_B = B S cos(\theta)
	\end{equation}

	fem, fuerza electromotriz
	\begin{equation}
		\epsilon = -\dfrac{d\Phi_B}{dt} = -N \dfrac{d\Phi_B}{dt}
	\end{equation}
	
	corriente I
	
	\begin{equation}
		I = \frac{\epsilon}{R}
	\end{equation}
	
\end{document}