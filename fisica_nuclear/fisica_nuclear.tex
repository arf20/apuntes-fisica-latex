\documentclass[12pt, letterpaper, twoside]{article}

\usepackage{amsmath}
\usepackage{siunitx}
\usepackage[version=4]{mhchem}

\title{Apuntes de Física Nuclear}
\author{Ángel Ruiz Fernández B2A}
\date{Abril 2023}

\usepackage[a4paper, total={6in, 9in}]{geometry}

\newcommand{\dd}[1]{\mathrm{d}#1}

\begin{document}
	\maketitle
	
	\section{Nucleo}
	Núclido son protones $p^+$ o neutrones $n$
	El numero atómico $A$ y másico $Z$ se conserva
	
	\begin{equation}
		\ce{_{Z}^{A}X}
	\end{equation}

	Numero de neutrones $n$
	\begin{equation}
		N = A - Z
	\end{equation}

	\section{Desintegración natural}
	Exulsados a velocidades relativistas
	\begin{itemize}
		\item Alpha $\alpha$ | \ce{ ^{A}_{Z}X -> ^{A-4}_{Z-2}Y^{2-} + ^{4}_{2}He^{2+}}
		\item Beta $\beta^{-|+}$ | \ce{^{A}_{Z}X -> ^{A}_{Z+1}Y^{1+} + ^{0}_{-1}e^-} o $e^+$
		
			\ce{^{1}_{0}n -> ^{1}_{1}p^{+} + ^{0}_{-1}e^{-} + $\bar{v_{e}}$}
		\item Gamma $\gamma$ | \ce{^{A}_{Z}X $^{\ast}$ -> ^{A}_{Z}X + \gamma}
	\end{itemize}

	\section{Desintegración Radioactiva}
	Actividad en Bq ($N_0$ numero de átomos)
	\begin{equation}
		A = \lambda N_0
	\end{equation}
	
	Núcleos desintegrados
	\begin{equation}
		N = N_0 \cdot e^{-\lambda t}
	\end{equation}

	Tiempo de desintegración de N núcleos
	\begin{equation}
		t = - \frac{ln(\frac{N}{N_0})}{\lambda}
	\end{equation}

	Periodo de semidesintegración
	\begin{equation}
		T_{1/2} = \frac{ln 2}{\lambda}
	\end{equation}

%	Energía de enlace de nuclidos
%\ce{	\begin{equation}%
%		\Delta m = m_p + m_n - m{^{2}_{1}H^{+}
%	\end{equation} }
$\tau$
\end{document}