\documentclass[12pt, letterpaper, twoside]{article}

\usepackage{amsmath}

\title{Apuntes de Gravitación y Electrostática}
\author{Ángel Ruiz Fernández B1A}
\date{Marzo 2022, Rev. 2}

\usepackage[a4paper, total={6in, 9in}]{geometry}

\newcommand{\dd}[1]{\mathrm{d}#1}

%\usepackage[showframe,paper=a4paper,margin=1in]{geometry}
%\setlength{\droptitle}{-0.5in}


\begin{document}
	\maketitle
	
	\section{Gravitación}
	
		\subsection{Leyes de Kepler}
		
		
			1. Los planetas se mueven en órbitas elípticas, en uno de cuyos focos se encuentra un cuerpo M.
			
			2. El radiovector que une un cuerpo m con M barre áreas iguales en tiempos iguales.
			
			\begin{equation}
				\frac{dA}{dt} = cte
			\end{equation}
			
			\begin{equation}
				dA = \frac{1}{2} r r d \theta =  \frac{1}{2} r^2 d \theta
			\end{equation}
			
			\begin{equation}
				\frac{dA}{dt} = \frac{1}{2} r^2 \frac{d\theta}{dt}
			\end{equation}
			
			3. Los cuadrados de los periodos orbitales de los planetas son proporcionales a los cubos de los
			semiejes mayores de sus órbitas
			
			\begin{equation}
				\frac{T^2}{a^3} = cte
			\end{equation}
			
			\begin{equation}
				\frac{a^3_1}{a^3_2} = \frac{T^2_1(M+m_1)}{T^2_2(M+m_2)}
			\end{equation}
	
		\subsection{Ley de Gravitación Universal}
		
			Fuerza
			
			\begin{equation}
				\vec{F} = m \vec{g}
			\end{equation}
			
			\begin{equation}
				\vec{F} = \frac{GMm}{r^2} \hat{u}_r
			\end{equation}
			
			Campo
			
			\begin{equation}
				\vec{g} = \frac{GM}{r^2} \hat{u}_r
			\end{equation}
			
			Energía potencial
			
			\begin{equation}
				E_{pg} = -\frac{GMm}{r}
			\end{equation}
			
			Potencial gravitatorio
			
			\begin{equation}
				V_{g} = -\frac{GM}{r}
			\end{equation}
			
		\subsection{Ley de Gauss para campo gravitatorio}
		
			
			
	\section{Electrostática}
	
		\subsection{Ley de Coulomb}
		
		
	
\end{document}