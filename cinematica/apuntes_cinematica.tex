\documentclass[12pt, letterpaper, twoside]{article}

\usepackage{amsmath}

\title{Apuntes de cinematica}
\author{Ángel Ruiz Fernández B1A}
\date{Marzo 2022, Rev. 2}

\usepackage[a4paper, total={6in, 9in}]{geometry}

\newcommand{\dd}[1]{\mathrm{d}#1}

%\usepackage[showframe,paper=a4paper,margin=1in]{geometry}
%\setlength{\droptitle}{-0.5in}


\begin{document}
	\maketitle
	
	Vector posición $\vec{r}$ definido a partir de sus componentes cartesianas $r_x$ e $r_y$, asumiendo un espacio cartesiano bidimensional
	
	\begin{equation}
		\vec{r} = r_x \hat{\imath} + r_y \hat{\jmath}
	\end{equation}
	
	Vector velocidad $\vec{v}$ a partir del vector posición $\vec{r}$
	
	\begin{equation}
		\vec{v} = \dfrac{d\vec{r}}{dt}
	\end{equation}
	
	Vector aceleración $\vec{a}$ a partir del vector velocidad $\vec{v}$
	
	\begin{equation}
		\vec{a} = \dfrac{d\vec{vº}}{dt}
	\end{equation}
	
	Escalar rapidez $v$ a partir del vector velocidad $\vec{v}$
	
	\begin{equation}
		v = \lvert \vec{v} \lvert = \sqrt{v_x^2 + v_y^2}
	\end{equation}
	
	Vector unitario tangencial a la trayectoria $\hat{u}_t$ a partir de velocidad $\vec{v}$ y rapidez $v$
	
	\begin{equation}
		\hat{u}_t = \frac{\vec{v}}{v}
	\end{equation}
	
	Vector aceleración tangencial $\vec{a}_t$ a partir de la rapidez $v$ y el vector unitario tangencial $\hat{u}_t$
	
	\begin{equation}
		\vec{a}_t = \dfrac{dv}{dt} \cdot \hat{u}_t
	\end{equation}
	
	Vector aceleración normal $\vec{a}_n$ perpendicular a la tangencial en dirección al centro de un circulo tangente a la trayectoria, a partir de su radio $R$ y la rapidez $v$. Y su relación con el vector aceleración $\vec{a}$ y el vector aceleración tangencial $\vec{a}_t$
	
	\begin{equation}
		\vec{a}_n = \frac{v^2}{R} \cdot \hat{u}_n = \vec{a} - \vec{a}_t
	\end{equation}
	
	Suerte my friends el día 9 de Marzo. Made with \LaTeX
\end{document}